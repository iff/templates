\documentclass[xcolor=pdftex,table,10pt]{beamer}

\usepackage{pgfpages}
%\pgfpagesuselayout{4 on 1}[a4paper, border shrink=5mm, landscape]
%\setbeamercolor{background canvas}{bg=black,fg=white}
%\usebeamercolor[fg]{background canvas}

\mode<presentation> {
  \usetheme{keynote-iff}
}

\usepackage[english]{babel}
\usepackage[latin1]{inputenc}
\usepackage{times}
\usepackage{listings}
\usepackage{colortbl}
\usepackage{tikz}
\usepackage{verbatim}
\usepackage{algorithm}
\usepackage{algorithmic}
\usepackage{amsmath}
\usepackage{amsfonts}
\usepackage{url}
\usetikzlibrary{shapes,arrows,snakes,backgrounds}
\usefonttheme[onlymath]{serif}

\title[SHORT]{TITLE}
\author[Yves Ineichen]{Yves Ineichen \\ \texttt{ineichen@inf.ethz.ch}}
\institute{Institute}
\date{7th May 2010}

\begin{document}

% For every picture that defines or uses external nodes, you'll have to
% apply the 'remember picture' style. To avoid some typing, we'll apply
% the style to all pictures.
\tikzstyle{every picture}+=[remember picture]

% By default all math in TikZ nodes are set in inline mode. Change this to
% displaystyle so that we don't get small fractions.
\everymath{\displaystyle}

\lstset{language=C++, basicstyle=\small}


\begin{frame}
    \titlepage
\end{frame}


\begin{frame}
\frametitle{Frametitle}

    \begin{block}{Block}
        bla bla bla...
    \end{block}

    \vspace{0.1cm}
    
    \begin{exampleblock}{ExampleBlock}
        \begin{itemize}
            \item aa
            \item bb
        \end{itemize}
    \end{exampleblock}
    
    \vspace{0.1cm}

    \begin{alertblock}{AlertBlock}  
    \begin{itemize}
        \item aa 
        \item bb
    \end{itemize}
    \end{alertblock}

\end{frame}

\end{document}
